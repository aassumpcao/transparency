
\begin{table}[!htbp] \centering
  \caption{The Effect of Active Transparency on Information Requests}
  \label{tab:transparency2}
\scriptsize
\begin{tabular}{@{\extracolsep{5pt}}lcccc}
\\[-1.8ex]\hline
\hline \\[-1.8ex]
& \multicolumn{4}{c}{Outcomes:} \T \B \\
\cline{2-5}
 & \multicolumn{2}{c}{FOI Request (time)} & \multicolumn{2}{c}{FOI Request (quality)} \T \B \\
\\[-1.8ex] & \multicolumn{1}{c}{(1)} & \multicolumn{1}{c}{(2)} & \multicolumn{1}{c}{(3)} & \multicolumn{1}{c}{(4)} \B \\
\hline \\[-1.8ex]
 Active Transparency & $-.073^{***}$ & $-.050^{***}$ & $-.085^{***}$ & $-.063^{***}$ \\
                     & (.004) & (.003) & (.004) & (.004) \\
                     & & & & \\
\hline \\[-1.8ex]
Municipal Controls & \multicolumn{1}{c}{-} & \multicolumn{1}{c}{Yes} & \multicolumn{1}{c}{-} & \multicolumn{1}{c}{Yes} \\
Year Fixed-Effects & \multicolumn{1}{c}{-} & \multicolumn{1}{c}{Yes} & \multicolumn{1}{c}{-} & \multicolumn{1}{c}{Yes} \\
\hline \\[-1.8ex]
Observations & \multicolumn{1}{c}{4,404} & \multicolumn{1}{c}{4,404} & \multicolumn{1}{c}{4,404} & \multicolumn{1}{c}{4,404} \\
R$^{2}$ & \multicolumn{1}{c}{.002} & \multicolumn{1}{c}{.122} & \multicolumn{1}{c}{.002} & \multicolumn{1}{c}{.123} \\
Adjusted R$^{2}$ & \multicolumn{1}{c}{.001} & \multicolumn{1}{c}{.119} & \multicolumn{1}{c}{.002} & \multicolumn{1}{c}{.120} \\
\hline
\hline \\[-1.8ex]
\multicolumn{5}{p{.6\textwidth}}{\emph{Note:} This table displays the regressions measuring the effect of active transparency (being audited by a team of officials from the Office of the Comptroller-General -- \emph{CGU}) on information requests for a random sample of municipalities across Brazil participating in the \emph{Transparent Brazil} program. For each outcome, I display two regressions including and excluding municipal controls and year fixed-effects. The variable of interest is whether the municipality was audited by CGU after 2012. Robust standard errors are in parentheses. $^{*}$p$<$0.1; $^{**}$p$<$0.05; $^{***}$p$<$0.01.}\\
\end{tabular}
\end{table}
